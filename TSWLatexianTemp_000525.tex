\documentclass[]{article}

\usepackage{fancyhdr}
\usepackage{extramarks}
\usepackage{amsmath}
\usepackage{amsthm}
\usepackage{amsfonts}
\usepackage{tikz}
\usepackage[plain]{algorithm}
\usepackage{algpseudocode}
\usetikzlibrary{arrows,intersections}
\usepackage{wrapfig}
\usepackage{listings}
\usepackage[utf8]{inputenc}

\usepackage[T1]{fontenc}
\usepackage[american]{babel}

\usepackage{graphicx}
\usepackage{fancyvrb}
\usepackage{listings}
\usepackage{bm}
\usepackage{xcolor}
\usepackage{graphicx}
%\graphicspath{ {home/Users/EricaJhLee/MSAN/LinearRegression/hw/hw1/} }

\xdefinecolor{gray}{rgb}{0.4,0.4,0.4}
\xdefinecolor{blue}{RGB}{58,95,205}% R's royalblue3; #3A5FCD

\lstset{% setup listings
	language=R,% set programming language
	basicstyle=\ttfamily\small,% basic font style
	keywordstyle=\color{blue},% keyword style
        commentstyle=\color{gray},% comment style
	numbers=left,% display line numbers on the left side
	numberstyle=\scriptsize,% use small line numbers
	numbersep=10pt,% space between line numbers and code
	tabsize=3,% sizes of tabs
	showstringspaces=false,% do not replace spaces in strings by a certain character
	captionpos=b,% positioning of the caption below
        breaklines=true,% automatic line breaking
        escapeinside={(*}{*)},% escaping to LaTeX
        fancyvrb=true,% verbatim code is typset by listings
        extendedchars=false,% prohibit extended chars (chars of codes 128--255)
        literate={"}{{\texttt{"}}}1{<-}{{$\bm\leftarrow$}}1{<<-}{{$\bm\twoheadleftarrow$}}1
        {~}{{$\bm\sim$}}1{<=}{{$\bm\le$}}1{>=}{{$\bm\ge$}}1{!=}{{$\bm\neq$}}1{^}{{$^{\bm\wedge}$}}1,% item to replace, text, length of chars
        alsoletter={.<-},% becomes a letter
        alsoother={$},% becomes other
        otherkeywords={!=, ~, $, \&, \%/\%, \%*\%, \%\%, <-, <<-, /},% other keywords
        deletekeywords={c}% remove keywords
}


\usetikzlibrary{automata,positioning}

%
% Basic Document Settings
%

\topmargin=-0.45in
\evensidemargin=0in
\oddsidemargin=0in
\textwidth=6.5in
\textheight=9.0in
\headsep=0.25in

\linespread{1.1}

\pagestyle{fancy}
%\lhead{\hmwkAuthorName}
%\chead{\hmwkClass\ (\hmwkClassInstructor\ \hmwkClassTime)}
%\rhead{\hmwkTitle}
%\lfoot{\lastxmark}
%\cfoot{\thepage}

\renewcommand\headrulewidth{0.4pt}
\renewcommand\footrulewidth{0.4pt}

\setlength\parindent{0pt}

%
% Create Problem Sections
%
%
% Homework Problem Environment
%
% This environment takes an optional argument. When given, it will adjust the
% problem counter. This is useful for when the problems given for your
% assignment aren't sequential. See the last 3 problems of this template for an
% example.
%
%
% Homework Details
%   - Title
%   - Due date
%   - Class
%   - Section/Time
%   - Instructor
%   - Author
%
%
%\newcommand{\tab}[1]{\hspace{.160\textwidth}\rlap{#1}}
%\newcommand{\hmwkTitle}{Homework\ \#1}
%\newcommand{\hmwkDueDate}{February 12, 2014}
%\newcommand{\hmwkClass}{Linear Regression}
%\newcommand{\hmwkClassInstructor}{ Professor Paul Intrevado}
%\newcommand{\hmwkAuthorName}{Erica Jh Lee}

%
% Title Page
%

\title{
    \vspace{2in}
    \textmd{\textbf{\hmwkClass:\ \hmwkTitle}}\\
    \vspace{0.1in}\large{\textit{\hmwkClassInstructor\ \hmwkClassTime}}
    \vspace{3in}
}

\author{\textbf{\hmwkAuthorName}}
\date{}

\renewcommand{\part}[1]{\textbf{\large Part \Alph{partCounter}}\stepcounter{partCounter}\\}

%
% Various Helper Commands
%

% Useful for algorithms
\newcommand{\alg}[1]{\textsc{\bfseries \footnotesize #1}}
% For derivatives
\newcommand{\deriv}[1]{\frac{\mathrm{d}}{\mathrm{d}x} (#1)}
% For partial derivatives
\newcommand{\pderiv}[2]{\frac{\partial}{\partial #1} (#2)}
% Integral dx
\newcommand{\dx}{\mathrm{d}x}
% Alias for the Solution section header
\newcommand{\solution}{\textbf{\large Solution}}
% Probability commands: Expectation, Variance, Covariance, Bias
\newcommand{\E}{\mathrm{E}}
\newcommand{\Var}{\mathrm{Var}}
\newcommand{\Cov}{\mathrm{Cov}}
\newcommand{\Bias}{\mathrm{Bias}}

\begin{document}

\title{Types of Random Variables (Models)}
\author{Erica Lee}
\date{March 31, 2016}
\maketitle


\section{Discrete Random Variables}


\subsection*{Bernoulli(p)}
Y has two possible outcomes "Success" or "Failure"\\

Y$\sim$ Bernoulli(p), p = P(``Success"), in other words, Y = \# success of 1 trial\\

\textbullet \ Bernoulli : Distribution\\
\indent \textbullet \ p : parameter that characterizes distribution\\

$\mathbf{E}[Y] = p$\\
\indent Var($Y$) = p(1-p)\\




\subsection*{Binomial(n,p)}
Y = \# of successes in n independent Bernoulli(p) trials\\

Y$\sim$ Binomial(n,p), p = P(``Success"), in other words, Y = \# success of 1 trial\\

$\mathbf{E}[Y] = np$\\
\indent Var($Y$) = np(1-p)\\

$X_1, \dot , X_n \sim$ iid Bermoulli(p), then, Y = $\sum^{n} X_i \sim Bin(n,p)$\\



\subsection*{Poisson($\lambda$)}
Y is a random variable representing a count\\

Y$\sim$ Po($\lambda$),\\ 
\indent Y $\in$ \{0,1,2\}\\
\indent$\mathbf{E}[Y] = Var($Y$) = \lambda$\\\\
Note: the only discrete RV this is true for...




\subsection*{Geometric(p)}

Y = \# of trials (with Binary outcomes) needed before the $1^{st}$ success\\
Y $\sim$ Geometric(p)\\
p = probability of success\\

Y $\in$ \{ 1,2,3, ...\}

$\mathbf{E}[Y] = \frac{1}{p}$\\
Var(Y) = $\frac{1-p}{p^2}$\\

\textit{Only one discrete distribution that is Memory-less!}
\subsubsection*{Memoryless Property}
A random variable Y is memoryless if P(Y>t|Y>s) = P(Y>t-s), t-s\\\\
\textbullet \ Discrete distribution: Geometric(p)\\
\textbullet \ Continuous distribution: Exponential($\lambda$)



\subsection*{Negative Binomial(p, k)}

Y = \# of trials needed before k success (trials have only "successes" or "failures")\\
Y $\sim$ NegtiveBinomial(k)\\
p = probability of success\\

Y $\in$ \{ k,k+1, ...\}

$\mathbf{E}[Y] = \frac{pr}{(1-p)}$\\
Var(Y) = $\frac{pr}{(1-p)^2}$\\



\section{Continuous Random Variables}
\subsection*{Uniform(a,b)}
Y has equal chance of taking on any interval $[a,b]$\\
p(Y $\in [x, x+l]$) = $\frac{l}{b-a}$
Y$\sim$ Uniform(a,b)
$\mathbf{E}[Y] = \frac{a+b}{2}$\\
Var(Y) = $\frac{(b-a)^2}{12}$\\

\subsection*{Normal($\mu, \sigma^2$)}
Y$\sim N(\mu,\sigma^2)$\\
if the density (histogram) of Y is bell-shaped\\

\subsection*{Exponential($\lambda$)}
Y = waiting time before an event occurs\\
Y $\sim \ \text{Exp}(\lambda)$ \\
$\lambda$ is a mean waiting time AND the rate\\
Var(Y) = $\lambda^2$



\end{document}